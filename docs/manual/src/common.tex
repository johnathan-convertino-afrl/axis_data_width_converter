\begin{titlepage}
  \begin{center}

  {\Huge AXIS\_DATA\_WIDTH\_CONVERTER}

  \vspace{25mm}

  \includegraphics[width=0.90\textwidth,height=\textheight,keepaspectratio]{img/AFRL.png}

  \vspace{25mm}

  \today

  \vspace{15mm}

  {\Large Jay Convertino}

  \end{center}
\end{titlepage}

\tableofcontents

\newpage

\section{Usage}

\subsection{Introduction}

\par
This data width converter is for even integer divides of slave to master or
master to slave. Example this core can go from 4 bytes to 2 bytes or 2 bytes to
4 bytes. It can not go from 5 bytes to 2 bytes or 2 bytes to 5 bytes. 4/2 is 2, a
round number. 5/2 is a fractional number that will not work with this core.

\subsection{Dependencies}

\par
The following are the dependencies of the cores.

\begin{itemize}
  \item fusesoc 2.X
  \item iverilog (simulation)
  \item cocotb (simulation)
\end{itemize}

\subsubsection{fusesoc\_info Depenecies}
\begin{itemize}
\item dep
	\begin{itemize}
	\item AFRL:utility:helper:1.0.0
	\end{itemize}
\item dep\_tb
	\begin{itemize}
	\item AFRL:simulation:axis\_stimulator
	\item AFRL:simulation:clock\_stimulator
	\item AFRL:utility:sim\_helper
	\end{itemize}
\end{itemize}


\subsection{In a Project}
\par
Simply use this core between a sink and source AXIS devices. This will convert from one BUS size to another. Check the code to see if others will work correctly.

\section{Architecture}
\par
The only module is the axis\_data\_width\_converter module. It is listed below.

\begin{itemize}
  \item \textbf{axis\_data\_width\_converter } Impliment an algorithm to convert BUS data interfaces in even mutlples (see core for documentation).
\end{itemize}

\par
This core only uses a combinatoral method to divide the accumulator. Since all weights are powers of two this is done with a part select based on bit position.

\par
The always block has the following steps.
\begin{enumerate}
\item If there is valid data, sum the new data into the accumulator and remove the top element in the buffer from the accumulator.
\item Insert the new element into the buffer.
\item Shift the buffer to so that old elements at the top of the buffer are shifted out.
\end{enumerate}

Please see \ref{Module Documentation} for more information.

\section{Building}

\par
The AXIS data width converter core is written in Verilog 2001. They should synthesize in any modern FPGA software. The core comes as a fusesoc packaged core and can be
included in any other core. Be sure to make sure you have meet the dependencies listed in the previous section.

\subsection{fusesoc}
\par
Fusesoc is a system for building FPGA software without relying on the internal project management of the tool. Avoiding vendor lock in to Vivado or Quartus.
These cores, when included in a project, can be easily integrated and targets created based upon the end developer needs. The core by itself is not a part of
a system and should be integrated into a fusesoc based system. Simulations are setup to use fusesoc and are a part of its targets.

\subsection{Source Files}

\subsubsection{fusesoc\_info File List}
\begin{itemize}
\item src
	\begin{itemize}
	\item src/axis\_data\_width\_converter.v
	\end{itemize}
\item tb
	\begin{itemize}
	\item {'tb/tb\_axis.v': {'file\_type': 'verilogSource'}}
	\end{itemize}
\end{itemize}


\subsection{Targets}

\subsubsection{fusesoc\_info Targets}
\begin{itemize}
\item default
	\begin{itemize}
	\item[$\space$] Info: Default for IP intergration.
	\end{itemize}
\item sim
	\begin{itemize}
	\item[$\space$] Info: Test 1:1 conversion.
	\end{itemize}
\item sim\_reduce
	\begin{itemize}
	\item[$\space$] Info: Test data reduction.
	\end{itemize}
\item sim\_rand\_data\_reduce
	\begin{itemize}
	\item[$\space$] Info: Test data reduction with random data
	\end{itemize}
\item sim\_rand\_ready\_rand\_data\_reduce
	\begin{itemize}
	\item[$\space$] Info: Test data reduction with random ready and random data.
	\end{itemize}
\item sim\_8bit\_count\_data\_reduce
	\begin{itemize}
	\item[$\space$] Info: Test data reduction with counter data.
	\end{itemize}
\item sim\_rand\_ready\_8bit\_count\_data\_reduce
	\begin{itemize}
	\item[$\space$] Info: Test data reduction with counter data, and random ready.
	\end{itemize}
\item sim\_increase
	\begin{itemize}
	\item[$\space$] Info: Test data increase.
	\end{itemize}
\item sim\_rand\_data\_increase
	\begin{itemize}
	\item[$\space$] Info: Test data increase with random data.
	\end{itemize}
\item sim\_rand\_ready\_rand\_data\_increase
	\begin{itemize}
	\item[$\space$] Info: Test data increase with random data, and random ready.
	\end{itemize}
\item sim\_8bit\_count\_data\_increase
	\begin{itemize}
	\item[$\space$] Info: Test data increase with count data.
	\end{itemize}
\item sim\_rand\_ready\_8bit\_count\_data\_increase
	\begin{itemize}
	\item[$\space$] Info: Test data increase with count data, and random ready.
	\end{itemize}
\end{itemize}


\subsection{Directory Guide}

\par
Below highlights important folders from the root of the directory.

\begin{enumerate}
  \item \textbf{docs} Contains all documentation related to this project.
    \begin{itemize}
      \item \textbf{manual} Contains user manual and github page that are generated from the latex sources.
    \end{itemize}
  \item \textbf{src} Contains source files for the core
  \item \textbf{tb} Contains test bench files for iverilog and cocotb
    \begin{itemize}
      \item \textbf{cocotb} testbench files
    \end{itemize}
\end{enumerate}

\newpage

\section{Simulation}
\par
There are a few different simulations that can be run for this core.

\subsection{iverilog}
\par
iverilog is used for simple test benches for quick verification, visually, of the core.

\subsection{cocotb}
\par
Future simulations will use cocotb. This feature is not yet implemented.

\newpage

\section{Module Documentation} \label{Module Documentation}

\par
There is a single async module for this core.

\begin{itemize}
\item \textbf{axis\_data\_width\_converter} AXIS data width converter, converts from one BUS data size to another.\\
\end{itemize}
The next sections document the module in great detail.

