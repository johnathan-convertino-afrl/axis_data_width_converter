\begin{titlepage}
  \begin{center}

  {\Huge AXIS\_DATA\_WIDTH\_CONVERTER}

  \vspace{25mm}

  \includegraphics[width=0.90\textwidth,height=\textheight,keepaspectratio]{img/AFRL.png}

  \vspace{25mm}

  \today

  \vspace{15mm}

  {\Large Jay Convertino}

  \end{center}
\end{titlepage}

\tableofcontents

\newpage

\section{Usage}

\subsection{Introduction}

\par
This data width converter is for even integer divides of slave to master or
master to slave. Example this core can go from 4 bytes to 2 bytes or 2 bytes to
4 bytes. It can not go from 5 bytes to 2 bytes or 2 bytes to 5 bytes. 4/2 is 2, a
round number. 5/2 is a fractional number that will not work with this core.

\subsection{Dependencies}

\par
The following are the dependencies of the cores.

\begin{itemize}
  \item fusesoc 2.X
  \item iverilog (simulation)
  \item cocotb (simulation)
\end{itemize}

\subsubsection{fusesoc\_info Depenecies}
\begin{itemize}
\item dep
	\begin{itemize}
	\item AFRL:utility:helper:1.0.0
	\end{itemize}
\item dep\_tb
	\begin{itemize}
	\item AFRL:simulation:axis\_stimulator
	\item AFRL:simulation:clock\_stimulator
	\item AFRL:utility:sim\_helper
	\end{itemize}
\end{itemize}


\subsection{In a Project}
\par
Simply use this core between a sink and source AXIS devices. This will convert from one BUS size to another. Check the code to see if others will work correctly.

\section{Architecture}
\par
The only module is the axis\_data\_width\_converter module. It is listed below.

\begin{itemize}
  \item \textbf{axis\_data\_width\_converter } Implement an algorithm to convert BUS data interfaces in even multiples (see core for documentation).
\end{itemize}

\par
The data width converter uses a generate block to select between three possible scenarios. First is they are equal, which just assigns the fields to each other.
The second is slave is smaller then the master. This uses a register build up method by building up data till the correct number of bytes is reached. The final
method is slave is larger than the master. This works by slicing the incoming data into chunks for the slave.

\par
The slave is smaller than master block has the following steps.
\begin{enumerate}
\item Create registers used to buffer data and signals
\item Generate a backpressure tready signal for the axis input using the current tready master and its previous state.
\item The output is valid if the register has valid.
\item Last is set if register has last and the data is valid.
\item always block processes data in the following manner.
  \begin{enumerate}
    \item Once out of reset, check if the output device is ready. If it is clear out the data tlast and valid registers and set the previous tready to 0.
    \item If the input is valid, and we were previously not ready or are currently ready start processing slave data.
    \begin{enumerate}
      \item build up slave data in buffer
      \item build up last buffer
      \item increment counter, or decrement if reversed byte order.
      \item Once counter hits its threshold, reset counter to initial value, and the buffer data is now valid so register for valid is set to active high.
    \end{enumerate}
  \end{enumerate}
\end{enumerate}

\par
The slave is larger than master block has the following steps.
\begin{enumerate}
\item Use a for loop to generate an assignment to take input data and slice it into a wire that is segmented into the size of the output data.
\item Ready happens when counter hits its count and the proper signals are set. Backpressure is needed since going larger to smaller will take more clock cycles.
\item For the output, if the register for valid is active high, output the proper signal or data.
\item always block processes data in the following manner.
  \begin{enumerate}
    \item Once out of reset, check if the output device is ready. If it is, register for valid is set to 0 and the previous tready is cleared.
    \item If the input is valid, and we were previously not ready or are currently ready, and the counter has reach the start count, start processing slave data.
    \begin{enumerate}
      \item In an unrolled for loop take the split input data and store it in a buffer.
      \item increment counter, or decrement if reversed byte order.
      \item Data is valid, so set it active high
      \item previous tready is set to active high, since the core has to be ready to take data.
    \end{enumerate}
    \item Check the counter, and check if the destination device is ready, if it is, decrement the counter and reassert the valid and previous tready.
  \end{enumerate}
\end{enumerate}

Please see \ref{Module Documentation} for more information.

\section{Building}

\par
The AXIS data width converter core is written in Verilog 2001. They should synthesize in any modern FPGA software. The core comes as a fusesoc packaged core and can be
included in any other core. Be sure to make sure you have meet the dependencies listed in the previous section.

\subsection{fusesoc}
\par
Fusesoc is a system for building FPGA software without relying on the internal project management of the tool. Avoiding vendor lock in to Vivado or Quartus.
These cores, when included in a project, can be easily integrated and targets created based upon the end developer needs. The core by itself is not a part of
a system and should be integrated into a fusesoc based system. Simulations are setup to use fusesoc and are a part of its targets.

\subsection{Source Files}

\subsubsection{fusesoc\_info File List}
\begin{itemize}
\item src
	\begin{itemize}
	\item src/axis\_data\_width\_converter.v
	\end{itemize}
\item tb
	\begin{itemize}
	\item {'tb/tb\_axis.v': {'file\_type': 'verilogSource'}}
	\end{itemize}
\end{itemize}


\subsection{Targets}

\subsubsection{fusesoc\_info Targets}
\begin{itemize}
\item default
	\begin{itemize}
	\item[$\space$] Info: Default for IP intergration.
	\end{itemize}
\item sim
	\begin{itemize}
	\item[$\space$] Info: Test 1:1 conversion.
	\end{itemize}
\item sim\_reduce
	\begin{itemize}
	\item[$\space$] Info: Test data reduction.
	\end{itemize}
\item sim\_rand\_data\_reduce
	\begin{itemize}
	\item[$\space$] Info: Test data reduction with random data
	\end{itemize}
\item sim\_rand\_ready\_rand\_data\_reduce
	\begin{itemize}
	\item[$\space$] Info: Test data reduction with random ready and random data.
	\end{itemize}
\item sim\_8bit\_count\_data\_reduce
	\begin{itemize}
	\item[$\space$] Info: Test data reduction with counter data.
	\end{itemize}
\item sim\_rand\_ready\_8bit\_count\_data\_reduce
	\begin{itemize}
	\item[$\space$] Info: Test data reduction with counter data, and random ready.
	\end{itemize}
\item sim\_increase
	\begin{itemize}
	\item[$\space$] Info: Test data increase.
	\end{itemize}
\item sim\_rand\_data\_increase
	\begin{itemize}
	\item[$\space$] Info: Test data increase with random data.
	\end{itemize}
\item sim\_rand\_ready\_rand\_data\_increase
	\begin{itemize}
	\item[$\space$] Info: Test data increase with random data, and random ready.
	\end{itemize}
\item sim\_8bit\_count\_data\_increase
	\begin{itemize}
	\item[$\space$] Info: Test data increase with count data.
	\end{itemize}
\item sim\_rand\_ready\_8bit\_count\_data\_increase
	\begin{itemize}
	\item[$\space$] Info: Test data increase with count data, and random ready.
	\end{itemize}
\end{itemize}


\subsection{Directory Guide}

\par
Below highlights important folders from the root of the directory.

\begin{enumerate}
  \item \textbf{docs} Contains all documentation related to this project.
    \begin{itemize}
      \item \textbf{manual} Contains user manual and github page that are generated from the latex sources.
    \end{itemize}
  \item \textbf{src} Contains source files for the core
  \item \textbf{tb} Contains test bench files for iverilog and cocotb
    \begin{itemize}
      \item \textbf{cocotb} testbench files
    \end{itemize}
\end{enumerate}

\newpage

\section{Simulation}
\par
There are a few different simulations that can be run for this core.

\subsection{iverilog}
\par
iverilog is used for simple test benches for quick verification, visually, of the core.

\subsection{cocotb}
\par
Future simulations will use cocotb. This feature is not yet implemented.

\newpage

\section{Module Documentation} \label{Module Documentation}

\par
There is a single async module for this core.

\begin{itemize}
\item \textbf{axis\_data\_width\_converter} AXIS data width converter, converts from one BUS data size to another.\\
\end{itemize}
The next sections document the module in great detail.

